\begin{abstract}
Automated Program Repair (APR) has been profoundly transformed by Large Language Models (LLMs), enabling powerful bug fixing and code generation capabilities. APR's fundamental goal is to \textbf{reduce the time and resources required to identify and fix bugs in code} \cite{dikici2025}. However, concurrency debugging remains a central challenge due to the inherent \textbf{non-deterministic} nature of multi-threaded programs \cite{chang2021} and the high prevalence of data races (accounting for approximately 80\% of all concurrency defects [Previous Turns, 296]). Effective APR in this domain requires models that can handle complex defects and mitigate the introduction of new bugs or code smells \cite{liyanage2025}.

This paper details an empirical investigation into how LLMs manage concurrent programming bugs as defect complexity increases, specifically focusing on multi-hunk repairs. Our analysis highlights significant scaling differences, demonstrating that the \textbf{Gemini-1.5-Pro} model exhibits superior ability to scale accuracy with complexity. While low-complexity 1-bug problems yielded an average accuracy of 73.41\%, performance dramatically improved on the most complex scenarios, reaching fixation rates of \textbf{98.52\%} for 4-bug defects and a remarkable \textbf{99.6\%} for 5-bug defects [Previous Turns]. Furthermore, Gemini demonstrated high patch integrity, showing a low rate of introducing new bugs, such as only 4 new bugs introduced across the 2-bug test cases, and only 1 in the 5-bug test cases [Previous Turns].

Crucially, we identify a fundamental limitation in competing models: LLMs derived from the GPT architecture struggle significantly to verify program correctness under \textbf{Relaxed Memory Models (RMMs)}, such as Total Store Order (TSO) and Partial Store Order (PSO) \cite{wolff2024, jain2025}. This weakness undermines semantic preservation, risking the generation of plausible but incorrect patches \cite{anand2024, yang2023}. To bridge this critical verification gap, we advocate for the future direction of \textbf{Analysis-Augmented Generation (AAG)}, integrating static and dynamic analysis tools specialized in concurrency (such as \textit{ConcBugAssist} for constraint solving diagnosis \cite{khoshnood2015}) directly into the LLM prompt, thereby transforming LLMs from high-level pattern fixers into robust and verifiably correct concurrency repair agents.
\end{abstract}